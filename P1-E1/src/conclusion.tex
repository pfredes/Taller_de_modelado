Como conclusión se puede decir que ProcessMaker, aunque no sigue la nomenclatura BPM 2.0  pues está ordientada a servicios SOAP y su implementación no es muy facil, pues además no  posee gran documentación,  es una herramienta robusta para utilizar, pues no solo permite modelar los procesos de negocio como flujo de trabajo sino tambien implementar, gestionar y procesar procesos y solicitudes. Por otro lado, al ser una aplicación web, garantiza la conexión y su uso desde cualquier parte del mundo solo con  una conexión a internet. Además permite un sistema dinamico de modelado de procesos, pues cualquier cambio no es dificil de agregar.
\\
Una herramienta que para medianas empresas, permitirá ordenar, y acelerar ciertas tareas dentro de la organización.