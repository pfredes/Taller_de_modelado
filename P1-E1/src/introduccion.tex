En el siguiente informe se hace un análisis en profundidad de la herramienta llamada ProcessMaker, que es capaz de modelar y diseñar procesos de una organización,  además de poseer variadas opciones que otorga al cliente (organización o empresa). Con esto la organización puede dar a los usuarios finales que pueden ser sus propios trabajadores, administradores, gerentes, etc. que puedan manejar los distintos procesos que se tienen y ver como se comportan de acuerdo al flujo que existe en la organización siendo participes de estos al mismo tiempo.

También se presentan 5 criterios con los que se evaluó esta herramienta, observando las cualidades y defectos que posee, con respecto a los software que utilizan comúnmente las organizaciones.