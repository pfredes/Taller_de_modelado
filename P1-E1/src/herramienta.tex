La herramienta que se utiliza para analizar es el software \textbf{``ProcessMaker"}.
\\
Desarrollado por Colosa, empresa Boliviana, es una herramienta que funciona con el modelo ASP (Application Service Provider), es decir la compañía ofrece la aplicación desde sus servidores por un pago mensual, o sea un alquiler.
ProcessMaker es ideal para diseñar y automatizar procesos que utilicen formularios y requieren de decisiones o aprobaciones y que actualmente están siendo manejados por email, hojas de excel, formularios en Word, o papeles.

A pesar de que funciona en forma semejante a un workflow, ProcessMaker contiene una funcionalidad más avanzada. ProcessMaker permite al usuario crear, modificar, e integrar aplicaciones directamente desde un navegador.

Con ProcessMaker los usuarios pueden crear aplicaciones que llenan y complementan funcionalidades faltantes en sistemas CRM, ERP, etc.

Además, ProcessMaker se integra sin dificultad a otros productos mediante una interfaz de Web Services.

ProcessMaker se presenta a los usuarios en una forma unificada, a pesar de que se estén usando diferentes sistemas internamente.

Los usuarios pueden usar inmediatamente plantillas (templates) prediseñadas de muchos procesos generales, o los pueden crear desde cero, según las necesidades de la empresa.

Está dirigido a pequeñas y medianas empresas o unidades estratégicas de negocio dentro de una organización mayor. Facilita a los gerentes de empresas sin experiencia en programación para crear modelos de procesos de negocio. Funciona con Windows y Linux, y da ayuda comercial
Una herramienta totalmente libre y de código abierto (Open Source), disponible para las pequeñas y medianas empresas que necesiten de una herramienta informática capaz de colaborar con las actividades y procesos que realizan.

\subsection{Componentes}
El sistema esta dividido en las siguientes componentes o modulos:
\begin{itemize}
\item  \textbf{Diseñador de flujos de trabajo}: Mediante esta componente los usuarios pueden manejar los diferentes procesos y flujos de la empresa, permitiendo diseñar distintos flujos de actividades necesarios para dar orden a los procesos en la organizacion.
Se puede añadir usuarios, documentos, mensajes y además formularios que darán consistencia al flujo de trabajo.

\item \textbf{Dynaforms}: Permite diseñar formularios personalizados para diferentes  procesos que realice la empresa. Permite administrar informacion desde fuentes externas a la aplicación modificando formularios y base de datos asociadas a cada proceso.

\item \textbf{Casos y Reportes}: Desde este modulo se puede efectuar seguimiento y ratreos de los distintos procesos elaborados y verificar si existe algun problema en su ejecución (retrasos, cuellos de botella, entre otros). Análisis de resultados con el fin de mejorar las tareas asociadas.
\end{itemize}

\subsection{Funcionalidad}
El usuario de ProcessMaker puede manejar los procesos y flujos predefinidos o generados por los administradores directamente desde su Navegador. También cada usuario tiene un dashboard personalizado, que muestra un resumen de todos los procesos y sus respectivos avances.

\subsection{Posibles áreas de uso}
ProcessMaker es una herramienta genérica que puede adaptarse a todo tipo de empresas. Por ejemplo, se la puede usar en las siguientes áreas:
\begin{itemize}
\item Administración y finanzas (compras, ventas, gastos).
\item Marketing y Ventas (Manejo y monitoreo de campañas de marketing, processos de ventas).
\item Recursos Humanos (procesos de selección, control de empleados, etc.).
\item Operaciones (Help Desks, RMA, etc.).
\end{itemize}

\subsection{Implementación}

¿Cómo implementarlo? Gracias a que es una herramienta libre, los costos quedan desechados, aunque para implementarlo deberemos de contar con algunos requisitos, necesarios para el buen funcionamiento de la aplicación.

Lo destacable es que la aplicación funciona bajo los Sistemas Operativos Windows, y además trabaja perfectamente en Linux; esto es posible, pues la aplicación no se instala en un ordenador local, sino que, se implementa en un servidor y hosting previamente contratado.

Gracias a que funciona por medio de un servidor web, su estructura de trabajo es desde un navegador, desde allí se podrán ejecutar los diferentes módulos y características antes mencionadas, posee un panel de control intuitivo y de fácil manejo para el usuario reciente. Cabe destacar que la aplicación esta disponible en el idioma español nativo, y bastara con visitar la página web oficial del proyecto para descubrir más herramientas disponibles.
Antes de descargar la aplicación, y ya contando con el servicio de hosting, debemos de establecer que este último sea compatible con: Apache 2.2.3, MySql versión 5.0.22, PHP versión 5.1.6. De no ser así, deberíamos consultar con el proveedor del servicio para cambiar las características compatibles con la aplicación.

En caso de no disponer de un servidor hosting, se puede contratar el servicio de Process Maker Live, que no es otra cosa que, la aplicación instalada en los servidores del proyecto, y que tiene un costo mensual muy competitivo; es opcional, pero es una opción a tener en cuenta.







