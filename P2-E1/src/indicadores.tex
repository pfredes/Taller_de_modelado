Indicadores...
uno de los indicadores puede ser la cantidad de solicitud de pan para la distribucion
con este indicador tenemos la posibilidad de ver que tan importante es la recepcion
de solicitud de pedidos por parte del cliente hacia el repartidor especifico, en caso de que sean muchos
se puede ver la posibilidad de modificar el proceso de recepcion de pedido, tratar que desde la parte tecnica
modificar algun micro proceso con tal de hacer mas agil, rapido y eficiente la toma de pedidos

Otro podria ser en la cantidad de pan vendido vs pan que sale de produccion hacia sala de ventas
con esto se puede ver si se necesita mas pan, ya sea para la sala de ventas o para produccion dependiendo
del dia. Se puede hacer un analisis con este indicador tratando de predecir la produccion exacta permitiendo
tambien que no se produscan perdidas de material o de pan. Tambien si es necesario mas pan en ciertos dias,
modificar el proceso con algun paralelismo con tal de hacer mas rapida las actividades.

Y por ultimo un indicador seria de acuerdo a la produccion predecir la cantidad de compra para abastecimiento
con esto la prediccion de materia prima para la produccion de los distintos productos de esta panaderia haria mas 
eficiente esta misma. Pudiendose a su vez eliminar algun proceso que este a la deriva por la poca exactitud 
de la compra de materiales dependiendo toda la tarea en el administrador. Persona que recibe multiples tareas
haciendo que tenga demasiadas cosas en mente perjudicando que la organizacion sea eficaz en algun Macro proceso.
