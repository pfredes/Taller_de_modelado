\begin{itemize}

\item \textbf{Cantidad de solicitud de pan para la distribución:}
\\
\\
Con este indicador se tiene la posibilidad de ver que tan importante es la recepci\'on de solicitud de pedidos por parte del cliente hacia el repartidor espec\'ifico, en caso de que sean muchos, es decir, la panadería se expanda hacia otros sectores de Rancagua o de la región, se puede analizar la posibilidad de modificar el proceso de recepci\'on de pedido e intentar que mediante tecnologías de la información modificar alg\'un  proceso con tal de hacer mas \'agil, r\'apido y eficiente la toma de pedidos. De acuerdo a esto, se puede medir, luego de optimizar, si la cantidad de solicitudes aumentan usando este tipo de herramientas.
\\
\item \textbf{Cantidad de pan vendido v/s pan que sale de producción hacia la sala de ventas:}
\\
\\
En este aspecto, se puede verificar si exite la necesidad de incrementar o disminuir la producci\'on de pan, ya sea para la sala de ventas o para la distribución a clientes, de acuerdo al dia y fecha en cuanto a demanda de pan. De acuerdo a esto, se puede comprobar las perdidas asociadas a la producción innecesaria y a raiz de esto mejorar el proceso de abastecimiento, a fin de eliminar costos in\'utiles.
\\
\item \textbf{Calidad del pan:}
\\
\\
La calidad del pan, medible de acuerdo a la satisfacción del cliente, sin duda es un aspecto importante para mantener a flote el negocio. Es por esto, que el proceso de producción de pan, debe ser ciudadosamente observado, a fin de mantener los estandares de calidad que la politica de la empresa posee. Para optimizar este seguimiento, se pueden utilizar varias herramientas que pueden ser incorporadas en un nuevo proceso pequeño de control de calidad. 
\\
\item \textbf{Tiempo de ejecución de las tareas:}
\\
\\
El tiempo de ejecución va directamente asociado al indicador de calidad del producto, en ese aspecto, vinculado al proceso de control de calidad, se podrían realizar procesos paralelos y en conjunto de tecnologías de la información agilizar el proceso de la panadería.

\end{itemize}
