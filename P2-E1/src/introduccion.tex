En este informe se realiza el modelado de los procesos de negocio de la panadería Rodenas que está ubicada en Rancagua, donde fabrica, vende y distribuye sus productos a lo largo de gran parte de la provincia del Cachapoal. Nace 1948 aproximadamente, en la localidad de Coya desde donde se expande hacia la ciudad de Rancagua, por la iniciativa e involucramiento de los hijos del fundador. En este momento va en la tercera generación de gestión, teniendo 20 trabajadores repartidos en distintos turnos, y otras panaderias con el mismo nombre pero sin relacion comercial con la panaderia en estudio, los lazos son meramente familiares.
Como misión manifiesta la de producir y vender pan de buen sabor, aspecto y duradero en el tiempo. Entregando productos de calidad en la región manteniendo la costumbre chilena en la fabricación de este.
La filosofía de la empresa con respecto a sus proyecciones o visión es bastante limitada. Manteniendo la postura de “vivir el día a día”, acortando la mirada a plazos lo más pequeño posible. De esta manera se ha mantenido más de 60 años funcionando, con todos los altos y bajos que ha manifestado la industria.
\\
Para efectos de este proyecto, se ha incorporado la sección ``repostería", la cual se encarga de producir tortas, pasteles y galletas. Cabe destacar que son procesos extraidos de la panadería del supermercado Hiper Lider 15 norte de la ciudad de Viña del mar. Si bien, la diferencia en cuanto a tamaño de ambos, es grande, el proceso que se debe hacer para su producción es el mismo. \\
Además es posible que este mecanismo también ayude a la empresa de panadería ampliar sus rangos de venta hacia la repostería.
