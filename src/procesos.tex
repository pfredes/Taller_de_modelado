\subsection{Actores y roles}
\begin{itemize}
\item \textbf{Administrador:} Es la cabeza de la empresa y el punto de control en el maximo de procesos que puede involucrarse, por esto la desiciones de la empresa pasan todas por él. Tambien se preocupa del abastecimiento de insumos y de la mantencion de los recursos. 

\item \textbf{Vendedoras:} En la sala de venta se preocupan de ser el primer contacto con los clientes, dandole accesos a los productos y cuantificandolos en dinero para que el cliente pueda pagar.

\item \textbf{Cajero:} La unica tarea de esta persona es cobrar el dinero en la sala  de venta. Siendo  la persona de confianza en la sala de venta del administrador, pero sin tener poder de desición en este lugar.

\item \textbf{Repartidores:} Se encargan de ser el nexo entre la panadería y la distribución de pan a mayor escala. Distribuyen el pan, cobran en algunos casos y tienen la potestad de poder hacer nuevos clientes de gran escala.

\item \textbf{Maestro Panadero:} Es el jefe en el salón de producción, el encargado de hacer el batido para la elaboración de pan y gestionar a los oficiales dentro de producción. 

\item \textbf{Oficiales:} Son los encargados de hacer las actividades más basicas en producción.

\item \textbf{Hornero:} Es el encargado de mantener el horno a una temperatura adecuada y de la cocción del pan.
\end{itemize}

\subsection{Modelos de procesos nivel 1}

\subsection{Descripción de modelos de procesos nivel 1} 
