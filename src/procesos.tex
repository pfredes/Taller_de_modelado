\subsection{Actores y roles}
\begin{itemize}
\item \textbf{Administrador:} Es la cabeza de la empresa y el punto de control en el maximo de procesos que puede involucrarse, por esto la desiciones de la empresa pasan todas por él. Tambien se preocupa del abastecimiento de insumos y de la mantencion de los recursos. 

\item \textbf{Vendedoras:} En la sala de venta se preocupan de ser el primer contacto con los clientes, dandole accesos a los productos y cuantificandolos en dinero para que el cliente pueda pagar.

\item \textbf{Cajero:} La unica tarea de esta persona es cobrar el dinero en la sala  de venta. Siendo  la persona de confianza en la sala de venta del administrador, pero sin tener poder de desición en este lugar.

\item \textbf{Repartidores:} Se encargan de ser el nexo entre la panadería y la distribución de pan a mayor escala. Distribuyen el pan, cobran en algunos casos y tienen la potestad de poder hacer nuevos clientes de gran escala.
\end{itemize}

\subsection{Modelos de procesos a nivel de Macro procesos}
La panadería Rodenas se dedica a la fabricación, venta y distribución de pan y reposteria. A nivel macro todo empieza con una solicitud diaria que da el inicio al trabajo, pasando por el proceso de fabricación de pan (Producción de pan) y el proceso de fabricación de reposteria (Producción de reposteria), una vez terminados estos procesos pueden pasar a la Distribución o Ventas dependiendo de que prioridad tiene cada uno en el mismo momento, esto pasa por una decisión del Administrador. Después tenemos el proceso Ventas que se encarga de la venta del pan, venta de resposteria y atención al cliente, pero puede suceder que el proceso de Ventas antes que termine se vea entorpecido por si falta o no pan o reposteria, aquí es donde vuelve al proceso de Producción y se realiza nuevamente el flujo. Paralelo a este proceso tenemos Distribución, que es el encargado de tomar los pedidos durante el día y terminar el flujo diario.

\begin{center}
\includegraphics[width=15cm]{./imagenes/Macro_proceso.png}\\
Figura 1: Macro-proceso de panadería Rodenas.
\end{center}

\subsection{Descripción de modelos de procesos nivel 1} 

\begin{itemize}

\item \textbf{Producción de pan:} Se divide en 3 etapas que se hacen en paralelo pero no están sincronizados, Hacer Marraqueta, Hacer Hallulla y Producir Otros.

\begin{center}
\includegraphics[width=13cm]{./imagenes/produccion_pan.png}\\
Figura 2: Proceso de producción de pan.
\end{center}

\item \textbf{Producción de reposteria:} Se divide en 3 etapas que se hacen en paralelo pero no están sincronizados, Producir tortas, Producir galletas, Producir pasteles.

\begin{center}
\includegraphics[width=13cm]{./imagenes/produccion_reposteria.png}\\
Figura 3: Proceso de producción de reposteria.
\end{center}

\item \textbf{Ventas:} El principal objetivo de este proceso es atender a la clientela, como punto de partida el cliente llega al local pidiendo el pan o reposteria que quiere, se Recibe su pedido, luego se verifica si hay pan o reposteria suficiente para atender lo requerido, puede haber o no, en caso de no haber pan o reposteria se ve cuanta cantidad se necesitaría aproximadamente y se envía a Producción, en este último pueden hacer más pan o reposteria o simplemente notifican que no se puede hacer más pan por motivos varios, en caso de hacer más pan o reposteria se envía pan o reposteria a la sala de ventas, se hace el vale. Ahora en caso de que haya pan o reposteria solamente se procede a hacer el vale. Luego el cliente recibe el vale con el cual procede a pagar finalizando el proceso.

\textbf{Actores:}
\begin{itemize}
\item Vendedora: La vendedora es la encargada de recibir al cliente que llega, recibe el pedido y revisa si hay stock de pan o reposteria en ese momento, en caso contrario pide en la fabrica más pan o reposteria, además es la encargada de hacer el vale para posteriormente el cliente pague.
\item Cajera: Es la encargada de cobrar el dinero al cliente
\item Cliente: El más importante dentro de este proceso, es quien hace el pedido y quien debe pagar por aquello además de recibir el vale por supuesto.
\end{itemize} 

\begin{center}
\includegraphics[width=15cm]{./imagenes/ventas.png}\\
Figura 3: Proceso de Ventas.
\end{center}

\item \textbf{Distribución:} Este proceso se encarga a grandes rasgos de enviar el pedido a los distintos locales ya sea en la zona como en los alrededores. Todo empieza por recibir el pedido, luego se aprueba la solicitud  para después ver si se acepta el pedido, aceptar el pedido es ver si realmente mandar la cantidad que pide el cliente en caso de aceptar se hacen las cargas, en caso contrario se modifica la cantidad (siempre es bajando aquella) y hacer las cargas, luego paralelamente se comprueban las cargas y se registran para luego proceder al reparto, una vez en el lugar se ve si se cobra o no, en caso de cobrar el cliente debe pagar la cantidad predefinida, en caso contrario es porque el cliente paga mensualmente el pan distribuido. Finalmente se recauda el dinero dando término al proceso.

\textbf{Actores:} 
\begin{itemize}
\item Administrador: Persona que ve si realmente se aprueba la solicitud del cliente, es capaz de modificar el pedido en caso que el cliente devuelva pan, también comprueba las cargas y registra el reparto para tener un control de aquello y finalmente recauda el dinero del reparto
\item Repartidor: Es quien recibe el pedido, hace las cargas, reparte a los distintos locales y a veces dependiendo de cual sea el caso le toca cobrar a los clientes directamente.
\item Cliente: Es aquel que tiene un negocio o local y necesita de pan para atender a su propia clientela, solicita un pedido y a la hora de llegar puede pagarlo en el momento o simplemente pagarlo a final de mes.
\end{itemize}

\begin{center}
\includegraphics[width=15cm]{./imagenes/Distribucion.png}\\
Figura 3: Proceso de Ventas.
\end{center}

\item \textbf{Abastecimiento:}
\end{itemize}