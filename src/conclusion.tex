En cuanto a las conclusiones podemos decir, que modelar los procesos de una panadería es especialmente ideal, puesto que posee claramente identificados los roles de cada participante de las actividades que se deben realizar.\\
Se observa que la mayor parte de los procesos tienen como responsable al administrador de la organización, por lo que como mejora, si es que su carga de obligaciones es muy grande, se podría delegar a otra persona algunos procesos siendo el administrador solo un actor de supervision.\\
Por otro lado, como lo describimos al comienzo, si se realizan los cambios y extensiones pertinentes, es posible ampliar el rango de ventas de la panaderia, añadiendo una sección repostería que permitiera vender tortas, galletas y/o pasteles, utilizando la metodología adjunta.\\
En cuanto a la herramienta utilizada, es bastante fácil de utilizar, pues contiene todos los elementos de la notación BPMN 2.0 y está especialmente diseñado para este fin, en contraste con la herramienta ProcessMaker, la cual es un poco menos intuitiva además de no incorporar todos los elementos necesarios para el modelamiento, pero si para gestionarlos.